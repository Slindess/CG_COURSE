\documentclass[a4paper, 14pt]{article}
\usepackage[T2A]{fontenc}  % Поддержка кириллицы
\usepackage[utf8]{inputenc}  % Кодировка utf-8
\usepackage[russian]{babel}  % Поддержка русского языка
\usepackage{setspace}  % Пакет для изменения межстрочного интервала
\usepackage{geometry}  % Пакет для настройки полей страниц


% Настройка полей документа
\geometry{
  left=30mm,  % Левое поле
  right=10mm,  % Правое поле
  top=20mm,  % Верхнее поле
  bottom=20mm  % Нижнее поле
}

% Настройка шрифтов
\usepackage{fontspec}
\setmainfont{Times New Roman}
\onehalfspacing

\setlength{\parindent}{1.5cm}  % Отступ в начале абзаца
%\setlength{\parskip}{0em}

\sloppy

\usepackage{indentfirst}

\usepackage{fancyhdr}
\pagestyle{fancy}
\fancyhf{}  % Очищаем текущие настройки стиля
\fancyfoot[C]{\thepage}  % Установка нумерации по центру внизу страницы
%\thispagestyle{empty} - уберет нумерацию на конкретной странице (можно использовать на титульнике).

\begin{document}

\section{Аналитическая часть}
\subsection{Алгоритмы удаления невидимых поверхностей или линий}
\subsubsection{Алгоритм Робертса}
Алгорит Робертса работает в объектном пространстве. В первую очередь он удаляет те грани и линии, которые экранируются самим объектом. После чего, оставшиеся грани выпуклого многранника сравниваются с оставщимися гранями другого выпуклого многогранника. Если же многогранник не является выпуклым, то его придется разбивать на выпуклые, что приводит к дополнительным затратам. 

Для удаления невидимых объектов в контексте "горного ландшафта" он может оказаться крайне не эффективным, так как ландшафт состоит из невыпуклых объектов.

\subsubsection{Метод Z-Буффера} 
Метод Z-Буффера работает в пространстве изображений. В нем используется буффер для запоминания глубины каждого видимого пикселя. При обработке нового пикселя его глубина сравнивается с глубиной, записанной в буффере. Основной недостаток - потребление большого количества памяти. 

\subsubsection{Трассировка лучей}
Из положения камеры испускается луч через каждый пиксель изображения. Каждый луч проверяется на пересечение с объектом 3D сцены. В точке пересечения определяется освещенность поверхности по параметрам. Для рассчета теней испускаются тестовые лучи из точки пересечения к источникам света, если тестовый луч блокируют другие объекты сцены, то точка затеняется по выбранным алгоритмам. 

Для визуализации горного ландшафта, алгоритм трассировки лучей может стать самым оптимальным алгоритмом, так как он позволяет достичь высокого уровня реалистичности.

\subsection{Источники освещения}
\subsubsection{Точечный}
Точечный источник испускает свет, интенсивность которого уменьшается по мере отдаления от источника, во всех направлениях из точки.

\subsubsection{Направленный}
Равномерно освещает сцену в одном направлении, то есть сцена находится в однородном потоке света. Так как предполагается, что такой поток можно создать источником, удаленным на растояние, сильно превыщающем масштабы сцены, то его интенсивность не уменьшается. Примером такого источника является солнечный свет.

\subsubsection{Прожектор}
Прожектор создает направленный конус или усеченную пирамиду света. Сам свет становится более интенсивным по мере приближения к источнику и центру конуса.


\subsection{Отражение}
\subsubsection{Диффузное отражение}
Прямое освещение объекта равномерным количеством света, взаимодействующего с поверхностью. После того как свет падает на объект, он отражается в зависимости от свойств поверхности объекта, а также угла падения света. Свет, падающий перпендикулярно, создает более яркое освещение (Это явление описывается законом Ламберта). Матовые поверхности (дерево, камень, штукатурка) отражают больше диффузионного света, чем глянцевые, в результате чего они выглядят более мягкими.
При таком отражении положение наблюдателя не имеет значения, так как диффузно отраженный свет рассеивается равномерно по всем направлениям.

\subsubsection{Отражение окружающего света}
Окружающий свет не имеет направления. Его интенсивность определяется войствами материалов поверхностей объектов, а именно их коэффициентами отражения окружающего света.
Так же окружающий свет не отбрасывает тени и не имеет какого-то конкретного источника.
Окружающий свет действует на поверхности объектов независимо от их ориентации. Он добавляет базовый уровень освещенности, чтобы избежать полной темноты на теневых участках. Окружающий свет не требует сложных вычислений, таких как рассчет направления света.
 
\subsubsection{Зеркальное отражение}
Зеркальное отражение создает яркие пятна на объектах, исходя из интенсивности коэффициента зеркального отражения поверхности, а так же угла падения.
Каждый материал имеет свой коэффициент отражения.

\subsection{Модель освещения}
Модели освещения используются для визуализации световых эффектов, где все рассчеты воспроизводятся примерно на основе законов физики. Главной целью использования модели освещения является вычисления цвета каждого пикселя на основе его освещенности.

\subsubsection{Модель Фонга}
Модель Фонга является локальной моделью освещения, в которой освещенность пикселя определяется источниками света.
С помощью модели Фонга можно рассчитывать освещенность полигональных моделей.
Так как модель Фонга учитывает не только диффузионное отражение и отражение окружающего света, но и зеркальное отражение, то с ее помощью можно будет визуализировать блеск снежных горных вершин.
Модель Фонга учитывает положение зрителя для определения зеркального отражения.

\subsubsection{Глобальная модель освещения}
Глобальная модель освещения учитывает лучи, не только выпущенные из источника, но и отраженные лучи от других объектов. В результате получаются более реалистичные изображения, создание которых требует больших затрат.

\subsection{Технологии освещения}
\subsubsection{Карты освещения}
Карты освещения используется для предварительного расчета освещения. Подходит для статичных и не зеркальных объектов. На этапе подготовки сцены для каждого объекта рассчитывается вклад всех источников света. Значение освещенности в конкретной точке сохраняется, после чего накладывается на объекты. Предварительно посчитанное освещение избавляет от необходимости рассчитывать свет для каждого кадра.
Традиционно сцены ограничены локальной моделью освещения, так как глобальная требует высоких затрат. С помощью карты освещения можно провести вычисления в глобальной модели освещения один раз, что позволит достигнуть высого уровня реалистичности изображения, при снижении затрат.

\end{document}

