\documentclass[a4paper, 14pt, unknownkeysallowed]{extreport}
\usepackage[T2A]{fontenc}  % Поддержка кириллицы
\usepackage[utf8]{inputenc}  % Кодировка utf-8
\usepackage[russian]{babel}  % Поддержка русского языка
\usepackage{setspace}  % Пакет для изменения межстрочного интервала
\usepackage{geometry}  % Пакет для настройки полей страниц
\usepackage{amsmath}
\usepackage{xcolor}

\usepackage{titlesec}
\titleformat{\section}
	{\normalsize\bfseries}
	{\thesection}
	{1em}{}
\titlespacing*{\chapter}{1.5cm}{-22pt}{10pt}
\titlespacing*{\section}{1.5cm}{\baselineskip}{\baselineskip}
\titlespacing*{\subsection}{1.5cm}{\baselineskip}{\baselineskip}

\usepackage{titlesec}
\titleformat{\chapter}{\Large\bfseries}{\thechapter}{20pt}{\Large\bfseries}
\titleformat{\section}{\Large\bfseries}{\thesection}{20pt}{\Large\bfseries}

\newcommand{\chaptercentered}[1]{
	% Вот тут установить центрирование
	\titleformat{\chapter}{\Large\bfseries\centering}{\thechapter}{20pt}{\Large\bfseries}
    \chapter*{\MakeUppercase{#1}} % Центрирование заголовка
    \addcontentsline{toc}{chapter}{#1} % Добавление главы в оглавление
    \markboth{#1}{#1} % Обновление колонтитулов
    % Убрать центрирование
    \titleformat{\chapter}{\Large\bfseries}{\thechapter}{20pt}{\Large\bfseries}
}

% Настройка полей документа
\geometry{
  left=30mm,  % Левое поле
  right=10mm,  % Правое поле
  top=20mm,  % Верхнее поле
  bottom=20mm  % Нижнее поле
}

% Настройка шрифтов
\usepackage{fontspec}
\setmainfont{Times New Roman}
\onehalfspacing

\setlength{\parindent}{1.5cm}  % Отступ в начале абзаца
%\setlength{\parskip}{0em}

\sloppy

\frenchspacing
\usepackage{indentfirst}

\usepackage{fancyhdr}
\pagestyle{fancy}
\fancyhf{}  % Очищаем текущие настройки стиля
\fancyfoot[C]{\thepage}  % Установка нумерации по центру внизу страницы
%\thispagestyle{empty} - уберет нумерацию на конкретной странице (можно использовать на титульнике).
\renewcommand{\headrulewidth}{0pt}


\usepackage{caption}
% Картинки
\usepackage{pgfplots}
\usetikzlibrary{datavisualization}
\usetikzlibrary{datavisualization.formats.functions}

\usepackage{graphicx}

% Команды добавления изображений
\usepackage{float}
% Команда добавления изображений с параметром ширины
\captionsetup[figure]{name={Рисунок}, justification=centering, singlelinecheck=true, labelsep=endash}
\newcommand{\imgscheme}[3][\textwidth]{ % Первый аргумент - ширина по умолчанию
    \begin{figure}[H]
        \centering
        \includegraphics[width=#1, height=200mm, keepaspectratio]{img/#2} % Используем переданную ширину
        \caption{#3}
        \label{img/#2}
    \end{figure}
}

\newcommand{\imggraph}[2]{
	\begin{figure}[H]
		\center{\includegraphics[width=\textwidth, height=200mm, width=170mm, keepaspectratio]{img/#1}}
		\caption{#2}
		\label{img/#1}
	\end{figure}
}


\usepackage{csvsimple}


\usepackage{svg}

% Переименуем
%\renewcommand{\contentsname}{Содержание}
\titleformat{\bibname}{\Large\bfseries}{\chapter}{20pt}{\Large\bfseries}


% Листинги кода
\usepackage{listings}
\usepackage{xcolor}


\lstdefinestyle{C}{
    language=C,                       % Указание языка программирования C
    basicstyle=\small\ttfamily,       % Стиль шрифта для текста
    keywordstyle=\color{black},        % Цвет для ключевых слов
    stringstyle=\color{black},          % Цвет для строк
    commentstyle=\color{black},        % Цвет для комментариев
    numbers=left,                     % Нумерация строк слева
    numberstyle=\tiny,                % Размер шрифта для номеров строк
    stepnumber=1,                     % Шаг между номерами строк
    numbersep=5pt,                    % Расстояние от номеров строк до кода
    frame=single,                     % Рамка вокруг листинга
    tabsize=2,                        % Размер табуляции
    captionpos=t,                     % Позиция заголовка (внизу)
    breaklines=true,                  % Перенос строк
    breakatwhitespace=true,           % Перенос строк по пробелам
    backgroundcolor=\color{white},    % Цвет фона
}

\makeatletter
\def\@biblabel#1{#1. }
\makeatother

\lstset{
	literate=
	{а}{{\selectfont\char224}}1
	{б}{{\selectfont\char225}}1
	{в}{{\selectfont\char226}}1
	{г}{{\selectfont\char227}}1
	{д}{{\selectfont\char228}}1
	{е}{{\selectfont\char229}}1
	{ё}{{\"e}}1
	{ж}{{\selectfont\char230}}1
	{з}{{\selectfont\char231}}1
	{и}{{\selectfont\char232}}1
	{й}{{\selectfont\char233}}1
	{к}{{\selectfont\char234}}1
	{л}{{\selectfont\char235}}1
	{м}{{\selectfont\char236}}1
	{н}{{\selectfont\char237}}1
	{о}{{\selectfont\char238}}1
	{п}{{\selectfont\char239}}1
	{р}{{\selectfont\char240}}1
	{с}{{\selectfont\char241}}1
	{т}{{\selectfont\char242}}1
	{у}{{\selectfont\char243}}1
	{ф}{{\selectfont\char244}}1
	{х}{{\selectfont\char245}}1
	{ц}{{\selectfont\char246}}1
	{ч}{{\selectfont\char247}}1
	{ш}{{\selectfont\char248}}1
	{щ}{{\selectfont\char249}}1
	{ъ}{{\selectfont\char250}}1
	{ы}{{\selectfont\char251}}1
	{ь}{{\selectfont\char252}}1
	{э}{{\selectfont\char253}}1
	{ю}{{\selectfont\char254}}1
	{я}{{\selectfont\char255}}1
	{А}{{\selectfont\char192}}1
	{Б}{{\selectfont\char193}}1
	{В}{{\selectfont\char194}}1
	{Г}{{\selectfont\char195}}1
	{Д}{{\selectfont\char196}}1
	{Е}{{\selectfont\char197}}1
	{Ё}{{\"E}}1
	{Ж}{{\selectfont\char198}}1
	{З}{{\selectfont\char199}}1
	{И}{{\selectfont\char200}}1
	{Й}{{\selectfont\char201}}1
	{К}{{\selectfont\char202}}1
	{Л}{{\selectfont\char203}}1
	{М}{{\selectfont\char204}}1
	{Н}{{\selectfont\char205}}1
	{О}{{\selectfont\char206}}1
	{П}{{\selectfont\char207}}1
	{Р}{{\selectfont\char208}}1
	{С}{{\selectfont\char209}}1
	{Т}{{\selectfont\char210}}1
	{У}{{\selectfont\char211}}1
	{Ф}{{\selectfont\char212}}1
	{Х}{{\selectfont\char213}}1
	{Ц}{{\selectfont\char214}}1
	{Ч}{{\selectfont\char215}}1
	{Ш}{{\selectfont\char216}}1
	{Щ}{{\selectfont\char217}}1
	{Ъ}{{\selectfont\char218}}1
	{Ы}{{\selectfont\char219}}1
	{Ь}{{\selectfont\char220}}1
	{Э}{{\selectfont\char221}}1
	{Ю}{{\selectfont\char222}}1
	{Я}{{\selectfont\char223}}1
}